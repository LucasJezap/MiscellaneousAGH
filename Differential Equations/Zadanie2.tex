\documentclass[12pt]{article}
\usepackage{mathtools}
\usepackage{amssymb}
\usepackage{amsthm}
\usepackage{pgfplots}
\usepackage{tikz}
\usetikzlibrary{calc}
\usepackage{polski}
\usepackage[utf8]{inputenc}
\usepackage{geometry}
\usepackage{amsmath}
\usepackage{gensymb}
\usepackage{mnsymbol}
\usepackage{graphicx}
\usepackage{textgreek}
\usepackage{float}
\usepackage{caption}
\begin{document}
\newgeometry{tmargin=2cm,bmargin=2cm,lmargin=2cm,rmargin=2cm}
\begin{center}
\Large Łukasz Jezapkowicz
\end{center}
\begin{center}
\Large Rozwiązanie zadania numer 2 
\end{center}
Niech $\Omega$ - zbiór zdefiniowany na slajdzie $14.5$ oraz $h\in{C}^2(\bar{\Omega}\to\mathbb{R})$. Teza:
\begin{center}
\Large $$-\sum_{j,k=1}^{n} D_j(a_{jk}*D_kh)*v = \sum_{j,k=1}^{n}a_{jk}D_kh*D_jv-\sum_{j,k=1}^{n}D_j(a_{jk}*D_khv)$$ \newline
DOWÓD:
\end{center} 
W dowodzie korzystam z twierdzenia o pochodnej iloczynu funkcji: \newline
Tw. (o pochodnej iloczynu funkcji) \newline
Niech $f,g\in{C}^1(\mathbb{R}\to\mathbb{R})$ (lub gdy $f,g\in{C}^1(A\subset\mathbb{R}\to\mathbb{R})$. Wtedy: \newline
\begin{center}
\Large $D(fg)=Df*g + f*Dg$
\end{center}
A zatem korzystając z powyższego twierdzenia: \newline
\begin{center}
\Large $D_j(a_{jk}*D_khv) = D_j(a_{jk}*D_kh)*v + a_{jk}D_kh*D_jv$
\end{center}
Teraz nakładając na powyższą równość sumowanie po $j,k$ od $1$ do $n$ :
\begin{center}
\Large $$\sum_{j,k=1}^{n}D_j(a_{jk}*D_khv) = \sum_{j,k=1}^{n} D_j(a_{jk}*D_kh)*v + \sum_{j,k=1}^{n}a_{jk}D_kh*D_jv$$
\end{center}
Po przeniesieniu szeregów na drugie strony:
\begin{center}
\Large $$-\sum_{j,k=1}^{n} D_j(a_{jk}*D_kh)*v = \sum_{j,k=1}^{n}a_{jk}D_kh*D_jv-\sum_{j,k=1}^{n}D_j(a_{jk}*D_khv)$$
\end{center}
\end{document}
